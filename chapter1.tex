\documentclass[a4paper,11pt,abstracton,hidelinks]{scrartcl}
\usepackage{dphil}
\addbibresource{refs.bib}


\title{
Chapter 1. General introduction
}


\author{}


\begin{document}
\renewcommand{\abstractname}{Summary}


\maketitle


%%%%%%%%%%%%%%%%%%%%%%%%%%%%%%%%%%%%%%%%%%%%%%%%%%%%%%%%%%%%%%%%%%%%%%%%%%%%%%%
%%%%%%%%%%%%%%%%%%%%%%%%%%%%%%%%%%%%%%%%%%%%%%%%%%%%%%%%%%%%%%%%%%%%%%%%%%%%%%%
\begin{abstract}


%%
In this chapter I provide an overview of the present situation regarding malaria control in sub-Saharan Africa, the critical role played by large-scale mosquito control programmes, and the current challenges of insecticide resistance.
%
I then introduce whole-genome sequencing and describe its potential role in malaria mosquito surveillance and accelerating efforts towards the development of new mosquito control tools.
%
These factors provided the background and motivation for the establishment of the \textit{Anopheles gambiae} 1000 Genomes (Ag1000G) Project, an international collaboration to study genetic variation among malaria-transmitting mosquitoes collected from natural populations across sub-Saharan Africa.
%
This thesis describes work I contributed to phase 1 of the Ag1000G Project.
%


\end{abstract}


%%%%%%%%%%%%%%%%%%%%%%%%%%%%%%%%%%%%%%%%%%%%%%%%%%%%%%%%%%%%%%%%%%%%%%%%%%%%%%%
%%%%%%%%%%%%%%%%%%%%%%%%%%%%%%%%%%%%%%%%%%%%%%%%%%%%%%%%%%%%%%%%%%%%%%%%%%%%%%%
\section{Malaria vector control in sub-Saharan Africa}


%%
Malaria is an infectious disease caused by eukaryotic parasites of the genus \textit{Plasmodium} and transmitted by blood-feeding mosquitoes of the genus \textit{Anopheles}.
%
The vast majority of malaria occurs in sub-Saharan Africa, with more than 200 million cases and 400,000 deaths annually \citep{WHO2019WMR}.
%
The most effective methods of controlling malaria prevent disease transmission by targeting the mosquito vector (malaria vector control).
%
Two methods of malaria vector control are widely used in public health programmes in Africa: mass distribution of long-lasting insecticidal bednets (LLINs)
\citep{Carnevale2019,Okumu2020} and indoor residual spraying of houses with insecticides (IRS) \citep{WHO2006IRS,Pluess2010,Choi2019}.
%
At the turn of the millenium, a concerted international campaign and partnership was launched with the goal of dramatically increasing the scale and scope of malaria vector control programmes in sub-Saharan Africa \citep{Nabarro1998}.
%
Since then, more than 2 billion LLINs have been distributed in total, exceeding 100 million LLINs annually \citep{AMP2020} with more than 40\% of the population at risk sleeping under an LLIN from 2014 onwards \citep{Bhatt2015,WHO2019WMR}.
%
IRS programmes are more targeted and reserved for focal areas of higher malaria transmission, but are nevertheless substantial, protecting more than 10\% of the population at risk in 2010, although declining to lower levels since \citep{Bhatt2015,WHO2019WMR,Tangena2020}. 
%
The impact of these vector control programmes has been substantial. 
%
The prevalence of \textit{Plasmodium falciparum} malaria infections in Africa halved between 2000 and 2015, averting approximately 663 million clinical cases, of which 68\% can be attributed to LLIN and IRS interventions \citep{Bhatt2015}.
%%


\begin{figure}[t!]
\begin{center}
\includegraphics*[width=1\linewidth,center]{artwork/chapter1/llins.png}
\end{center}
\caption{
%
\textbf{Long-lasting insecticidal bednets (LLINs) distributed in sub-Saharan Africa}.
%
Bar colours denote countries, ordered vertically by the total number of nets distributed since 2000, largest at the bottom.
%
The eight countries with the largest numbers of nets are: Nigeria (dark blue); Democratic Republic of the Congo (light blue); Ethiopia (dark orange); Kenya (light orange); Uganda (dark green); Tanzania (light green); Ghana (dark red); Madagascar (light red).
%
Data from \citet{AMP2020}.
%
}
\label{fig:llins}
\end{figure}


%%%%%%%%%%%%%%%%%%%%%%%%%%%%%%%%%%%%%%%%%%%%%%%%%%%%%%%%%%%%%%%%%%%%%%%%%%%%%%%
%%%%%%%%%%%%%%%%%%%%%%%%%%%%%%%%%%%%%%%%%%%%%%%%%%%%%%%%%%%%%%%%%%%%%%%%%%%%%%%
\section{The threat of insecticide resistance}


%%
All LLINs approved for public health use by the World Health Organization (WHO) use a pyrethroid insecticide \citep{Elliott1989} as the primary active ingredient \citep{WHO2020PQVC}. 
%
The majority of countries with IRS programmes have also primarily used pyrethroid insecticides \citep{WHO2019WMR,Tangena2020}.
%
The heavy reliance on a single insecticide class has inevitably led to a substantial increase in the prevalence of resistance to pyrethroid insecticides among African malaria vectors \citep{Hemingway2016}. 
%
Pyrethroid resistance is conventionally monitored via standardised bioassays, and resistance is detected when the proportion of mosquitoes killed by a diagnostic dose falls below a standard threshold \citep{WHO2018TPIRM}. 
%
Many countries routinely perform these insecticide resistance bioassays as part of entomological monitoring programmes. 
%
When these bioassay data have been aggregated, there is a clear trend of increasing pyrethroid resistance since 2000 within the major African malaria vector species \citep{Hancock2020}. 
%
This trend is particularly acute in West Africa where, for example, it is estimated that more than 80\% of the region has resistance to the pyrethroid deltamethrin since 2010. 
%%


%%
Resistance to pyrethroid insecticides in malaria vectors clearly has the potential to reduce the efficacy of LLIN and IRS programmes.
%
In practice, however, it has not been straightforward to demonstrate or measure this reduction in efficacy.
%
A WHO-coordinated observational study in four African countries and India found no evidence that pyrethroid resistance had any effect on malaria infection prevalence or disease incidence in areas using pyrethroid LLINs as the primary intervention \citep{Kleinschmidt2018}.
%
On the other hand, several studies have found that switching IRS programmes from pyrethroids to a different insecticide class has led to a significant reduction in disease \citep{Hargreaves2000,Kafy2017}. 
%
Using non-pyrethroid IRS in addition to pyrethroid LLINs, or using newer LLINs that combine a pyrethroid with the synergist piperonyl butoxide (PBO) that counteracts some forms of pyrethroid resistance, have also been shown to reduce disease relative to pyrethroid-only LLINs alone \citep{Protopopoff2018}. 
%
Mathematical modelling of aggregated data from bioassays and experimental hut trials has also provided evidence that non-pyrethroid IRS substantially reduces malaria relative to pyrethroid IRS \citep{SherrardSmith2018} and that PBO LLINs are more effective than pyrethroid-only LLINs in areas with pyrethroid resistance \citep{Churcher2016}.
%
There is a growing consensus that pyrethroid resistance now poses a significant threat to malaria vector control programmes, and is at least in part responsible for the fact that malaria prevalence has not substantially reduced in sub-Saharan Africa since 2014 and has increased in some countries \citep{Hemingway2016,WHO2019WMR}. 
%%


%%%%%%%%%%%%%%%%%%%%%%%%%%%%%%%%%%%%%%%%%%%%%%%%%%%%%%%%%%%%%%%%%%%%%%%%%%%%%%%
%%%%%%%%%%%%%%%%%%%%%%%%%%%%%%%%%%%%%%%%%%%%%%%%%%%%%%%%%%%%%%%%%%%%%%%%%%%%%%%
\section{Insecticide resistance management}


%%
Recognising this trend of rising pyrethroid resistance and anticipating the threat to malaria vector control, the WHO developed a global plan for insecticide resistance management (GPIRM) \citep{WHO2012GPIRM}. 
%
The ultimate goal of insecticide resistance management (IRM) is to maintain the effectiveness of insecticides by preventing or delaying the evolution and spread of resistance mutations in mosquito populations. 
%
IRM principles and strategies have been developed and established in agricultural pest management, and these have largely formed the basis for malaria vector IRM recommendations \citep{Georghiou2005,Sternberg2018}. 
%
In general, there are three main IRM strategies:
%
\begin{enumerate}
%
\item \textbf{Management by moderation} - Reducing insecticide use by using them in conjunction with a range of non-insecticide-based control measures. For malaria vectors, this is also known as integrated vector management (IVM), and includes measures such as larval source management.
%
\item \textbf{Management by saturation} - Ensure that insecticides are delivered in a way that overwhelms the insect's natural defenses. For malaria vectors, this includes using synergists such as PBO that inhibit resistance mechanisms within LLINs, and  IRS formulations that use microencapsulation to increase the dose received on contact.
%
\item \textbf{Management by multiple attack} - Employ multiple insecticides with different modes of action. Specific strategies available for malaria vectors include LLINs impregnated with two different insecticides; IRS formulations that are mixtures of two different insecticides; and IRS programmes that regularly rotate treatments between different insecticides. This can also include deployment of both LLINs and IRS in the same location using different insecticides.
%
\end{enumerate}
%
These strategies form pillar \RN{1} of the GPIRM.
%%


%%
When the GPIRM was published in 2012, there were practical limitations on the range of possible IRM strategies for malaria vector control, because the only LLIN products available all used a single pyrethroid active ingredient, and a limited range of IRS products were available spanning four insecticide classes and two modes of action.
%
More recently, a ``next-generation'' of dual active ingredient LLIN products have been approved for public health use, including PBO LLINs \citep{Gleave2018} and LLINs that combine a pyrethroid with a second insecticide with a different mode of action \citep{Bayili2017,Tiono2018}.
%
A ``next-generation'' of IRS products have also become available, including a microencapsulation formulation of the organophosphate insecticide pyrimiphos methyl \citep{Oxborough2014}, and products using the neonicotinoid insecticide clothianidin not previously used in public health \citep{Oxborough2019}.
%
These new products have opened up new possibilities for malaria vector IRM.
%
Malaria control programmes began rotating IRS programmes to use next-generation IRS products from 2016 \citep{Tangena2020} and several initiatives are working to accelerate the deployment of next-generation LLINs\footnote{http://unitaid.org/call-for-proposal/catalyzing-market-introduction-next-generation-long-lasting-insecticidal-nets-llins/}$^{,}$\footnote{https://www.ivcc.com/market-access/new-nets-project/}.
%
These changes represent the most significant upheaval in malaria vector control strategy since the turn of the millenium, and present both new opportunities and new challenges.
%
In particular, the use of new insecticides will inevitably exert new selective pressures on malaria vector populations, and the emergence and spread of new forms of resistance is a major concern.
%
Also, next-generation LLIN and IRS products are more expensive and can be logistically more demanding \citep{TenBrink2018}.
%
Thus choices need to be made about where, when and how to deploy these new products to get maximum impact from limited resources \citep{WHO2017PBOLLIN}.
%%


\section{Surveillance as a core intervention}


%%
@@TODO
%%


\printbibliography


\end{document}
