\begin{refsection}
\chapter{General introduction}\label{ch:1}


%%%%%%%%%%%%%%%%%%%%%%%%%%%%%%%%%%%%%%%%%%%%%%%%%%%%%%%%%%%%%%%%%%%%%%%%%%%%%%%
%%%%%%%%%%%%%%%%%%%%%%%%%%%%%%%%%%%%%%%%%%%%%%%%%%%%%%%%%%%%%%%%%%%%%%%%%%%%%%%
\begin{abstract}


%%
In this chapter I provide an overview of the present situation regarding malaria control in sub-Saharan Africa, the critical role played by large-scale mosquito control programmes, and the current challenges of insecticide resistance.
%
I then introduce whole-genome sequencing and describe its potential role in malaria mosquito surveillance and accelerating efforts towards the development of new mosquito control tools.
%
These factors provided the background and motivation for the establishment of the \textit{Anopheles gambiae} 1000 Genomes (Ag1000G) Project, an international collaboration to study genetic variation among malaria-transmitting mosquitoes collected from natural populations across sub-Saharan Africa.
%
This thesis describes work I contributed to phase 1 of the Ag1000G Project.
%


\end{abstract}


%%%%%%%%%%%%%%%%%%%%%%%%%%%%%%%%%%%%%%%%%%%%%%%%%%%%%%%%%%%%%%%%%%%%%%%%%%%%%%%
%%%%%%%%%%%%%%%%%%%%%%%%%%%%%%%%%%%%%%%%%%%%%%%%%%%%%%%%%%%%%%%%%%%%%%%%%%%%%%%
\section[Malaria vector control]{Malaria vector control in sub-Saharan Africa}\label{sec:vector-control}


%%
Malaria is an infectious disease caused by eukaryotic parasites of the genus \textit{Plasmodium} and transmitted by blood-feeding mosquitoes of the genus \textit{Anopheles}.
%
The vast majority of malaria occurs in sub-Saharan Africa, with more than 200 million cases and 400,000 deaths annually~\parencite{WHO2019WMR}.
%
The most effective methods of controlling malaria prevent disease transmission by targeting the mosquito vector (malaria vector control).
%
Two methods of malaria vector control are widely used in public health programmes in Africa: mass distribution of long-lasting insecticidal bednets (LLINs)~\parencite{Carnevale2019,Okumu2020} and indoor residual spraying of houses with insecticides (IRS)~\parencite{WHO2006IRS,Pluess2010,Choi2019}.
%
At the turn of the millennium, a concerted international campaign and partnership was launched with the goal of dramatically increasing the scale and scope of malaria vector control programmes in sub-Saharan Africa~\parencite{Nabarro1998}.
%
Since then, more than 2 billion LLINs have been distributed in total, exceeding 100 million LLINs annually~\parencite{AMP2020} (Fig.~\ref{fig:llins}) with more than 40\% of the population at risk sleeping under an LLIN from 2014 onwards~\parencite{Bhatt2015,WHO2019WMR}.
%
IRS programmes are more targeted and reserved for focal areas of higher malaria transmission, but are nevertheless substantial, protecting more than 10\% of the population at risk in 2010, although declining to lower levels since~\parencite{Bhatt2015,WHO2019WMR,Tangena2020}. 
%
The impact of these vector control programmes has been substantial. 
%
The prevalence of \textit{Plasmodium falciparum} malaria infections in Africa halved between 2000 and 2015, averting approximately 663 million clinical cases, of which 68\% can be attributed to LLIN and IRS interventions~\parencite{Bhatt2015}.
%%


\begin{figure}[t!]
\begin{center}
\includegraphics*[width=1\linewidth,center]{artwork/chapter1/llins}
\end{center}
\caption{
%
\textbf{Long-lasting insecticidal bednets (LLINs) distributed in sub-Saharan Africa}.
%
Bar colours denote countries, ordered vertically by the total number of nets distributed since 2000, largest at the bottom.
%
The eight countries with the largest numbers of nets are: Nigeria (dark blue); Democratic Republic of the Congo (light blue); Ethiopia (dark orange); Kenya (light orange); Uganda (dark green); Tanzania (light green); Ghana (dark red); Madagascar (light red).
%
Data from \textcite{AMP2020}.
%
}
\label{fig:llins}
\end{figure}


%%%%%%%%%%%%%%%%%%%%%%%%%%%%%%%%%%%%%%%%%%%%%%%%%%%%%%%%%%%%%%%%%%%%%%%%%%%%%%%
%%%%%%%%%%%%%%%%%%%%%%%%%%%%%%%%%%%%%%%%%%%%%%%%%%%%%%%%%%%%%%%%%%%%%%%%%%%%%%%
\section{The threat of insecticide resistance}\label{sec:insecticide-resistance}


%%
All LLINs approved for public health use by the World Health Organization (WHO) use a pyrethroid insecticide~\parencite{Elliott1989} as the primary active ingredient~\parencite{WHO2020PQVC}. 
%
The majority of countries with IRS programmes have also primarily used pyrethroid insecticides~\parencite{WHO2019WMR,Tangena2020}.
%
The heavy reliance on a single insecticide class has inevitably led to a substantial increase in the prevalence of resistance to pyrethroid insecticides among African malaria vectors~\parencite{Hemingway2016}. 
%
Pyrethroid resistance is conventionally monitored via standardised bioassays, and resistance is detected when the proportion of mosquitoes killed by a diagnostic dose falls below a standard threshold~\parencite{WHO2018TPIRM}. 
%
Many countries routinely perform these insecticide resistance bioassays as part of entomological monitoring programmes. 
%
When these bioassay data have been aggregated, there is a clear trend of increasing pyrethroid resistance since 2000 within the major African malaria vector species~\parencite{Hancock2020}. 
%
This trend is particularly acute in West Africa where, for example, it is estimated that more than 80\% of the region has resistance to the pyrethroid deltamethrin since 2010. 
%%


%%
Resistance to pyrethroid insecticides in malaria vectors clearly has the potential to reduce the efficacy of LLIN and IRS programmes.
%
In practice, however, it has not been straightforward to demonstrate or measure this reduction in efficacy.
%
A WHO-coordinated observational study in four African countries and India found no evidence that pyrethroid resistance had any effect on malaria infection prevalence or disease incidence in areas using pyrethroid LLINs as the primary intervention~\parencite{Kleinschmidt2018}.
%
On the other hand, several studies have found that switching IRS programmes from pyrethroids to a different insecticide class has led to a significant reduction in disease~\parencite{Hargreaves2000,Kafy2017}. 
%
Using non-pyrethroid IRS in addition to pyrethroid LLINs, or using newer LLINs that combine a pyrethroid with the synergist piperonyl butoxide (PBO) that counteracts some forms of pyrethroid resistance, have also been shown to reduce disease relative to pyrethroid-only LLINs alone~\parencite{Protopopoff2018}. 
%
Mathematical modelling of aggregated data from bioassays and experimental hut trials has also provided evidence that non-pyrethroid IRS substantially reduces malaria relative to pyrethroid IRS~\parencite{SherrardSmith2018} and that PBO LLINs are more effective than pyrethroid-only LLINs in areas with pyrethroid resistance~\parencite{Churcher2016}.
%
There is a growing consensus that pyrethroid resistance now poses a significant threat to malaria vector control programmes, and is at least in part responsible for the fact that malaria prevalence has not substantially reduced in sub-Saharan Africa since 2014 and has increased in some countries~\parencite{Hemingway2016,WHO2019WMR}. 
%%


%%%%%%%%%%%%%%%%%%%%%%%%%%%%%%%%%%%%%%%%%%%%%%%%%%%%%%%%%%%%%%%%%%%%%%%%%%%%%%%
%%%%%%%%%%%%%%%%%%%%%%%%%%%%%%%%%%%%%%%%%%%%%%%%%%%%%%%%%%%%%%%%%%%%%%%%%%%%%%%
\section{Insecticide resistance management}\label{sec:IRM}


%%
Recognising this trend of rising pyrethroid resistance and anticipating the threat to malaria vector control, the WHO developed a global plan for insecticide resistance management (GPIRM)~\parencite{WHO2012GPIRM}. 
%
The ultimate goal of insecticide resistance management (IRM) is to maintain the effectiveness of insecticides by preventing or delaying the evolution and spread of resistance mutations in mosquito populations. 
%
IRM principles and strategies have been developed and established in agricultural pest management, and these have largely formed the basis for malaria vector IRM recommendations~\parencite{Georghiou2005,Sternberg2018}. 
%
In general, there are three main IRM strategies:
%
\begin{enumerate}
%
\item \textbf{Management by moderation}.
%
Reducing insecticide use by using them in conjunction with a range of non-insecticide-based control measures.
%
For malaria vectors, this is also known as integrated vector management (IVM), and includes measures such as larval source management.
%
\item \textbf{Management by saturation}.
%
Ensure that insecticides are delivered in a way that overwhelms the insect's natural defenses.
%
For malaria vectors, this includes using synergists such as PBO that inhibit resistance mechanisms within LLINs, and  IRS formulations that use microencapsulation to increase the dose received on contact.
%
\item \textbf{Management by multiple attack}.
%
Employ multiple insecticides with different modes of action.
%
Specific strategies available for malaria vectors include LLINs impregnated with two different insecticides, IRS formulations that are mixtures of two different insecticides, and IRS programmes that regularly rotate treatments between different insecticides.
%
This can also include deployment of both LLINs and IRS in the same location using different insecticides.
%
\end{enumerate}
%
These strategies form pillar \RN{1} of the GPIRM.
%%


%%
When the GPIRM was published in 2012, there were practical limitations on the range of possible IRM strategies for malaria vector control, because the only LLIN products available all used a single pyrethroid active ingredient, and a limited range of IRS products were available spanning four insecticide classes and two modes of action.
%
More recently, a ``next-generation'' of dual active ingredient LLIN products have been approved for public health use, including PBO LLINs~\parencite{Gleave2018} and LLINs that combine a pyrethroid with a second insecticide with a different mode of action~\parencite{Bayili2017,Tiono2018}.
%
A ``next-generation'' of IRS products have also become available, including a microencapsulation formulation of the organophosphate insecticide pyrimiphos methyl~\parencite{Oxborough2014}, and products using the neonicotinoid insecticide clothianidin not previously used in public health~\parencite{Oxborough2019}.
%
These new products have opened up new possibilities for malaria vector IRM.
%
Malaria control programmes began rotating IRS programmes to use next-generation IRS products from 2016~\parencite{Tangena2020} and several initiatives are working to accelerate the deployment of next-generation LLINs\footnote{http://unitaid.org/call-for-proposal/catalyzing-market-introduction-next-generation-long-lasting-insecticidal-nets-llins/}$^{,}$\footnote{https://www.ivcc.com/market-access/new-nets-project/}.
%
These changes represent the most significant upheaval in malaria vector control strategy since the turn of the millennium, and present both new opportunities and new challenges.
%
In particular, the use of new insecticides will inevitably exert new selective pressures on malaria vector populations, and the emergence and spread of new forms of resistance is a major concern.
%
Also, next-generation LLIN and IRS products are more expensive and can be logistically more demanding~\parencite{TenBrink2018}.
%
Thus, choices need to be made about where, when and how to deploy these new products to get maximum impact from limited resources without reducing the number of people protected from malaria infection~\parencite{WHO2017PBOLLIN}.
%%


%%%%%%%%%%%%%%%%%%%%%%%%%%%%%%%%%%%%%%%%%%%%%%%%%%%%%%%%%%%%%%%%%%%%%%%%%%%%%%%
%%%%%%%%%%%%%%%%%%%%%%%%%%%%%%%%%%%%%%%%%%%%%%%%%%%%%%%%%%%%%%%%%%%%%%%%%%%%%%%
\section{Surveillance as a core intervention}\label{sec:surveillance}


%%
Data are thus needed to guide decisions regarding optimal vector control strategies in different settings, to evaluate the success and impact of interventions and resistance management strategies, and to provide early warning of new evolutionary adaptations in response to the deployment of next-generation LLIN and IRS products.
%
Recognising this need, the WHO Global Technical Strategy for Malaria 2016--2030 advocates that malaria surveillance should be transformed into a core intervention as one of three pillars of the strategy~\parencite{WHO2015GTS}.
%
Here ``surveillance'' covers a broad range of data gathering activities, encompassing both epidemiological and entomological variables, in addition to intervention coverage, available resources, trends in health service utilisation, and more. 
%
Vector surveillance, also known as entomological monitoring, is a critical component of this, gathering data on malaria vector populations.
%
Vector surveillance programmes are established to a varying degree in most malaria-endemic countries and typically collect data on which malaria vector species are present, their abundance and seasonality, their behaviour in terms of time and location of biting and host preference, as well as the susceptibility of each vector species to different insecticides~\parencite{Russell2020}.
%
The GPIRM emphasizes the need for insecticide resistance monitoring as pillar \RN{2} of its action plan~\parencite{WHO2012GPIRM}.
%%


%%
Current malaria vector surveillance programmes have two key limitations.
%
The first is that surveillance programmes are not sufficiently resourced, and thus there is a lack of trained personnel and facilities~\parencite{Russell2020}.
%
The second is that surveillance programmes generally do not have the capability to gather genetic data on malaria vector populations, and where they do the data are limited in resolution.
%
These data gaps mean that a number of key operational questions cannot be effectively answered, including but not limited to the following:
\begin{enumerate}
%
\item \textbf{Detecting cryptic vector species}. 
%
Malaria vector species occur within cryptic species complexes, where morphological identification is not sufficient to resolve species identity~\parencite{Davidson1964,Coetzee2013}. 
%
Failure to differentiate known species or recognise the presence of previously unknown species means that other data variables become muddled, because different species within the same complex may have different behaviours and/or insecticide resistance adaptations.
%
\item \textbf{Differentiating molecular mechanisms of insecticide resistance}.
%
The presence of insecticide resistance can be detected by standard WHO or CDC bioassays, but these do not provide information regarding the underlying molecular mechanisms of resistance that are present in a given malaria vector population~\parencite{WHO2018TPIRM}.
%
Given that a number of new vector control products such as PBO LLINs are designed to target a specific mechanism of resistance, the absence of these data means there is no way to determine whether such a product should be deployed.
%
\item \textbf{Providing early warning of novel insecticide resistance adaptations}.
%
There is limited ability to obtain early warning of novel insecticide resistance adaptations, such as those emerging in response to deployment of a new IRS or LLIN products, because resistance typically has to reach a relatively high frequency within a mosquito population before it becomes evident via conventional bioassays~\parencite{Roush1986,Sternberg2018}.
%
\item \textbf{Monitoring of insecticide resistance allele frequencies}.
%
IRM strategies such as IRS rotation depend on managing the frequency of insecticide resistance alleles within mosquito populations, switching products at the right time and using them for the right duration to ensure resistance does not reach fixation~\parencite{South2018}. 
%
However, without any data on resistance allele frequencies, there is no way to know if IRM strategies are working as intended.
%
\item \textbf{Tracking the spread of insecticide resistance}. 
%
Mosquitoes move and have the capability to spread insecticide resistance over the course of multiple generations of movement and reproduction~\parencite{Service1997,Huestis2019}.
%
Without genetic data, it is not possible to resolve whether resistance has originated locally or spread from elsewhere.
%
Thus, it is hard to determine which interventions or conditions are driving the emergence of resistance, and whether IRM strategies can be designed locally or need to be coordinated nationally or even internationally to be effective.
%
\end{enumerate}
%
An awareness of these gaps, coupled with the rapid advancement of high throughput technologies for genome sequencing and other molecular diagnostics, and the development and maturation of genomic epidemiology as a field of research, has driven an interest in the development of molecular methods and particularly genome sequencing for malaria vector surveillance.
%%


%%%%%%%%%%%%%%%%%%%%%%%%%%%%%%%%%%%%%%%%%%%%%%%%%%%%%%%%%%%%%%%%%%%%%%%%%%%%%%%
%%%%%%%%%%%%%%%%%%%%%%%%%%%%%%%%%%%%%%%%%%%%%%%%%%%%%%%%%%%%%%%%%%%%%%%%%%%%%%%
\section{New vector control tools are needed}\label{sec:new-tools}

 
%%
Even if next-generation LLIN and IRS products are widely deployed, IRM strategies are implemented, and these are supported by improved vector surveillance capabilities, there is a consensus that this will not be sufficient to reach malaria elimination.
%
New vector control tools will be required, and the need for expanding research to accelerate the development of new tools is recognised as pillar \RN{3} of the GPIRM~\parencite{WHO2012GPIRM} and as supporting element 1 of the WHO Global Technical Strategy for Malaria 2016--2030~\parencite{WHO2015GTS}.
%
This includes further repurposing of agricultural insecticides, as well as the development of new insecticides with novel modes of action designed specifically for public health use~\parencite{Hemingway2006,Lees2019}.
%
This also includes the development novel control tools that do not rely on insecticides, such as genetic control tools~\parencite{Davidson1974,Burt2003}.
%
In particular, CRISPR/Cas9 gene editing technology had enabled the development of highly effective gene drives, which are selfish genetic elements with the capability to spread through mosquito populations via super-Mendelian inheritance and cause population suppression or modification~\parencite{Burt2003,Kyrou2018}.
%
The development of these novel vector control tools has benefited greatly from and continues to rely upon high throughput molecular tools such as whole-genome sequencing, and the availability of high quality open access molecular data resources such as reference genome sequences for malaria vectors~\parencite{Holt2002,Sharakhova2007,Lawniczak2010,Neafsey2015}.
%
Increasingly, the value of and need for data on natural genetic variation in targeted mosquito populations is also becoming evident.
%
For example, CRISPR/Cas9 gene drives need to target regions of low genetic diversity to minimise the evolution of resistance~\parencite{Kyrou2018}.
%%


%%%%%%%%%%%%%%%%%%%%%%%%%%%%%%%%%%%%%%%%%%%%%%%%%%%%%%%%%%%%%%%%%%%%%%%%%%%%%%%
%%%%%%%%%%%%%%%%%%%%%%%%%%%%%%%%%%%%%%%%%%%%%%%%%%%%%%%%%%%%%%%%%%%%%%%%%%%%%%%
\section[Next-generation sequencing]{Next-generation sequencing as a tool for genomic epidemiology and infectious disease surveillance}\label{sec:NGS}


%%
High throughput ``next-generation'' genome sequencing (NGS) technologies have advanced rapidly in the last two decades, opening up new applications for the study and control of infectious diseases~\parencite{Goodwin2016}.
%
The fundamental innovation of NGS is to perform sequencing of up to millions of small fragments of DNA in parallel.
%
For example, in 2005 the Illumina Genome Analyzer was the first commercially available NGS instrument, capable of generating 1 gigabase (Gb) of data per run.
%
The Illumina HiSeq instruments available from 2011 extended this capability by two orders of magnitude, generating up to 600 Gb per run~\parencite{Illumina2017}.
%
These leaps in data generating capability, coupled with the corresponding reduction in per-unit sequencing costs, allow for the sequencing of many individuals of a given species, and thus the study of genetic variation within and between natural populations.
%
The 1000 Genomes Project pioneered the use of NGS for the study of genetic variation, sequencing the genomes of 2,504 individuals from 26 human populations, and making the data openly available as a resource for the research community~\parencite{1000G2015}.
%
Applied to infectious diseases, NGS allows pathogen genomes from many individual infections to be sequenced and compared, providing opportunities to discover genetic variation underlying important traits such as virulence or drug resistance~\parencite{Armstrong2019}.
%
Comparing genome sequences also reveals patterns of relatedness between pathogens in different hosts, which can be used to investigate outbreaks of bacterial and viral diseases, or the transmission dynamics of endemic parasite diseases such as malaria~\parencite{Robinson2013,Daniels2015,Wohl2016,Neafsey2017,Wesolowski2018,Armstrong2019}.
%%

%%
To promote the application of genomics to malaria research, the Malaria Genomic Epidemiology Network (MalariaGEN)\footnote{https://www.malariagen.net} was established in 2005, and provides a framework for equitable collaboration and data sharing between multiple research centres and public health laboratories with access to different sampling and sequencing capabilities. 
%
The first large-scale study of genome variation in malaria parasites carried out by MalariaGEN sequenced 227 samples from three continents and discovered 86 thousand polymorphisms within gene coding regions~\parencite{Manske2012}.
%
Subsequent studies expanded this dataset and applied it to identify multiple populations of drug resistant parasites in Cambodia and discover genetic variants associated with antimalarial drug resistance~\parencite{Miotto2013,Miotto2015}.
%
The most recent data release includes high quality genotype calls on 3 million single nucleotide polymorphisms (SNPs) and short insertion/deletions (indels) in 7,000 worldwide samples of P. falciparum infection~\parencite{MalariaGEN2019PF}.
%
Given these successes with use of NGS to study infectious disease pathogens, it is natural to ask whether a similar approach could be used to study disease vectors such as \textit{Anopheles} mosquitoes that transmit malaria.
%


%%%%%%%%%%%%%%%%%%%%%%%%%%%%%%%%%%%%%%%%%%%%%%%%%%%%%%%%%%%%%%%%%%%%%%%%%%%%%%%
%%%%%%%%%%%%%%%%%%%%%%%%%%%%%%%%%%%%%%%%%%%%%%%%%%%%%%%%%%%%%%%%%%%%%%%%%%%%%%%
\section{Genomic epidemiology of malaria vectors: opportunities and challenges}\label{sec:genomic-epi}


%%
The genome sequence of the major Afrotropical malaria vector species, \textit{Anopheles gambiae}, was sequenced in 2002 and updated in 2007~\parencite{Holt2002,Sharakhova2007}.
%
This reference sequence, scaffolded to complete chromosomes, spans a total of 230 Mb across the two autosomes and the X sex chromosome, with an additional 42.6 Mb in unplaced contigs.
%
The availability of a high quality reference sequence makes possible the analysis of natural genome variation via NGS, because short reads from individual mosquitoes can be aligned to the reference genome and variants identified by comparison with the reference sequence.
%
Many questions could be investigated using such data.
%
For example, although the general mechanisms of insecticide resistance in malaria vectors are reasonably well established, the specific genes and genetic variants underlying these resistance mechanisms are for the most part unknown.
%
Analogous to the discovery of drug resistance variants in malaria parasites or antimicrobial resistance variants in bacteria, NGS could be used to discover and build a more complete picture of the genetic basis of insecticide resistance in malaria vectors~\parencite{Donnelly2016}.
%
Analogous to the analysis of pathogen outbreaks, NGS could be used to analyse the geographical origins and movements of different insecticide resistance mutations between malaria vector populations.
%
The analysis of genetic variation can also be used to make demographic inferences about populations, including patterns of migration and changes in population size.
%
Of particular relevance to malaria vectors would be investigating demographic changes in response to vector control interventions.
%
Many of the gaps in vector surveillance data described above could in principle be filled by NGS of malaria vectors.
%
Open data resources of natural genetic variation could also accelerate many forms of research, including the research and development of new vector control tools.
%%


%%
Although there are parallels between genomic epidemiology of pathogens and vectors, there are also fundamental differences which present unique challenges, including:
%
\begin{enumerate}
%
\item \textbf{Genome size}.
%
Whereas viral genomes are typically less than 50 kb, bacterial genomes less than 5 Mb, and eukaryotic pathogen genomes such as \textit{Plasmodium} less than 30 Mb, \textit{Anopheles} genomes are an order of magnitude larger at \textasciitilde 300 Mb in size~\parencite{Neafsey2015}.
%
\item \textbf{Genome complexity}.
%
Viral genomes typically have 10--100 genes, bacteria 100--5,000 genes, \textit{Anopheles gambiae} has 13,057 annotated protein-coding genes~\parencite{AgamP4.12,GiraldoCalderon2015}.
%
The \textit{Anopheles gambiae} genome also has a diversity of repetitive elements including various classes of transposable element, not generally found in viruses or bacteria or eukaryotic parasites~\parencite{Tu2004,FernandezMedina2011}.
%
\item \textbf{Sexual reproduction and frequent recombination}.
%
Viral and bacterial pathogens are not sexually reproducing, and although some do undergo a form of recombination under certain circumstances, it is much less frequent than that which occurs in sexually reproducing eukaryotes.
%
The recombination rate of \textit{Anopheles gambiae} is estimated at ~1 cM/Mb, approximately one recombination event per chromosome per generation~\parencite{Pombi2006}.
%
\item \textbf{Haplodiploidy}. 
%
\textit{Anopheles gambiae} has a haplodiploid sex determination system, where sperm and unfertilized eggs are haploid and individuals are diploid, in common with other dipteran insects, mammals and various other eukaryotic phyla.
%
This means that sequencing the genomic DNA from a single individual provides information about both of the genome sequences present within that diploid individual.
%
\item \textbf{Large effective population size and high levels of genetic diversity}.
%
Previous studies of diversity within \textit{Anopheles} species based on sequencing of individual genes have found nucleotide diversity at synonymous coding sites in the range 0.4-2.9 \% depending on species, with \textit{Anopheles gambiae} being the highest among these~\parencite{Leffler2012}.
%
This is indicative of very large effective population size and is at the extreme end of diversity found across the tree of life.
%
\end{enumerate}
%
These factors require that analytical methods need to be established and validated that are appropriate to malaria vectors, in order to provide a solid foundation for epidemiological inferences to be made.
%%


%%%%%%%%%%%%%%%%%%%%%%%%%%%%%%%%%%%%%%%%%%%%%%%%%%%%%%%%%%%%%%%%%%%%%%%%%%%%%%%
%%%%%%%%%%%%%%%%%%%%%%%%%%%%%%%%%%%%%%%%%%%%%%%%%%%%%%%%%%%%%%%%%%%%%%%%%%%%%%%
\section{The \textit{Anopheles gambiae} 1000 Genomes Project}\label{sec:ag1000g}


%%
In 2013, the cost of Illumina whole-genome deep sequencing of individual \textit{Anopheles} mosquitoes had reached the level at which sequencing upwards of 1000 individuals was a feasible goal.
%
MalariaGEN, supported by the Malaria Programme at the Wellcome Trust Sanger Institute, established a project to sequence the genomes of mosquitoes in the \textit{Anopheles gambiae} species complex, collected from natural populations across sub-Saharan Africa.
%
To support this venture, a consortium was established, bringing together representatives from 20 different research institutions, including groups with expertise in field sampling, malaria vector biology and population genomics.
%
The project was named ``The \textit{Anopheles gambiae} 1000 Genomes Project'', abbreviated to ``Ag1000G''.
%
The primary goal of the Ag1000G Project was to generate a high quality open access data resource on natural genetic variation within \textit{Anopheles gambiae} populations, and to make these data available to the research and public health communities.
%
The Ag1000G Project also aimed to use these data to investigate the demography and evolution of \textit{Anopheles gambiae} populations, particularly in relation to malaria vector control interventions.
%
The ultimate aim was to establish a foundation comprising both a high quality data resource and a collection of proven analytical methods, that could then be used as a platform to develop and scale up NGS as an operational tool for malaria vector surveillance.
%

%%
For practical purposes, the Ag1000G Project was divided into three phases:
%
\begin{itemize}
%
\item \textbf{Ag1000G phase 1} sequenced 765 individual mosquitoes sample from natural populations in 8 African countries, with representation of two major malaria vector species within the \textit{Anopheles gambiae complex}, \textit{Anopheles gambiae} and \textit{Anopheles coluzzii}.
%
Primary analyses of those data were published in \textcite{Ag1000G2017}.
%
\item \textbf{Ag1000G phase 2} expanded the resource to include 1,142 wild-caught mosquitoes from 13 countries, with increased representation of \textit{Anopheles coluzzii}.
%
Primary analyses were published recently in \textcite{Ag1000G2020}.
%
\item \textbf{Ag1000G phase 3} is expanding the resource to include 2,784 wild-caught mosquitoes from 19 countries, with representation of a third vector species, \textit{Anopheles arabiensis}.
%
Analyses of those data are ongoing, with data scheduled for release in early 2021.
%
\end{itemize}



%%
My role within the Ag1000G Consortium is to coordinate the project as a whole, and to lead data production, curation and analysis.
%
Although this work has been carried out in collaboration with other members of the Ag1000G Consortium, there are specific areas where I made individual contributions, and those are the focus of this thesis.
%
In particular, this thesis documents the contributions I made during first project phase.
%%


\section{Structure of this thesis}\label{sec:structure}


%%
In the next chapter I take a brief historical detour, in order to introduce the \textit{Anopheles gambiae} species complex, which is the biological subject of this thesis.
%
Whilst working on the Ag1000G project I was fortunate to come into possession of a collection of previously unpublished letters between entomologists George Davidson and Hugh Paterson, documenting their collaboration during the 1960s which led to the discovery of the \textit{Anopheles gambiae} complex.
%
Their discovery marked the introduction of genetic methods into malaria entomology, and their correspondence provides a unique historical perspective on present malaria elimination efforts, as well as the potential role for genomic epidemiology in future vector control programmes.
%%

%%
In the third chapter I describe the production of the Ag1000G phase 1 genome variation data resource, which comprises high quality data on 52 million single nucleotide polymorphisms (SNPs).
%
The aim of the chapter is to describe robust methodologies for genome-wide discovery and genotyping of SNPs from Illumina high throughput deep sequencing of individual \textit{Anopheles} mosquitoes, and to evaluate the accuracy and sensitivity of the data produced in Ag1000G phase 1 using those methods.
%
The chapter concludes by describing the overall levels of genetic diversity within the Ag1000G phase 1 cohort as a whole, and provides confirmation that \textit{Anopheles gambiae} mosquitoes are indeed among the most genetically diverse organisms in the natural world.
%%


%%
In the fourth chapter I analyse the Ag1000G phase 1 data resource to investigate genetic population structure among wild-caught mosquitoes.
%
I also compare genetic diversity within these populations and quantify genetic differentiation between them.
%
Ag1000G phase 1 sampled mosquito populations from a broad geographical range, crossing the entire span of continental Africa from coast to coast, and there are many contrasts between the ecosystems inhabited by \textit{Anopheles gambiae} mosquitoes across this range, including coastal to inland locations, deep forest to savannah, and urban to rural settings.
%
The aim of the chapter is to uncover heterogeneities among the mosquito populations sampled across these diverse settings, and to begin to form an understanding of how major geographical and ecological features have influenced the shape, size and connectivity of contemporary populations.
%%


%%
In the fifth chapter I continue the analysis of the Ag1000G phase 1 resource, performing genome-wide scans for signals of recent positive selection.
%
The aim of the chapter is to identify genes experiencing the most extreme selection pressures, and thus which are likely to be involved in the adaptive response to the scale-up of vector control interventions within the last two decades.
%
I discover strong signals of selection at six functionally-validated insecticide resistance genes, providing confirmation that these genes are the central drivers of insecticide resistance in these mosquito populations.
%
I also describe the design and implementation of a web resource to allow exploration and fine-mapping of novel selection signals, and illustrate its use to investigate a novel selection signal at a diacylglycerol kinase gene, a novel candidate with potential links to insecticide resistance.
%%


%%
In the sixth chapter I focus on a single gene encoding the voltage-gated sodium channel protein, which is the physiological binding target of pyrethroid insecticides, and where amino acid substitutions have been shown to cause target-site insensitivity to pyrethroids.
%
I describe the discovery of novel non-synonymous substitutions within the \textit{Vgsc} gene, and present population-genetic evidence that these substitutions are unlikely to be neutral and should be studied further for evidence of an insecticide resistance phenotype.
%
I also use haplotype data from Ag1000G phase 1 to analyse the genetic backgrounds on which known resistance alleles are found, and make inferences about the spread of resistance alleles between different countries and mosquito species.
%%


%%
In the final chapter I look forward to future applications of genome sequencing within operational vector surveillance programmes, exploring the potential pathway to translation. 
%%


\printbibliography[
heading=subbibintoc,
title={References}
]

\end{refsection}
