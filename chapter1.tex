\documentclass[a4paper,11pt,abstracton,hidelinks]{scrartcl}
\usepackage{dphil}
\addbibresource{refs.bib}


\title{
Chapter 1. General introduction
}


\author{}


\begin{document}
\renewcommand{\abstractname}{Summary}


\maketitle


%%%%%%%%%%%%%%%%%%%%%%%%%%%%%%%%%%%%%%%%%%%%%%%%%%%%%%%%%%%%%%%%%%%%%%%%%%%%%%%
%%%%%%%%%%%%%%%%%%%%%%%%%%%%%%%%%%%%%%%%%%%%%%%%%%%%%%%%%%%%%%%%%%%%%%%%%%%%%%%
\begin{abstract}


%%
In this chapter I provide an overview of the present situation regarding malaria control in sub-Saharan Africa, the critical role played by large-scale mosquito control programmes, and the current challenges of insecticide resistance.
%
I then introduce whole-genome sequencing and describe its potential role in malaria mosquito surveillance and accelerating efforts towards the development of new mosquito control tools.
%
These factors provided the background and motivation for the establishment of the \textit{Anopheles gambiae} 1000 Genomes (Ag1000G) Project, an international collaboration to study genetic variation among malaria-transmitting mosquitoes collected from natural populations across sub-Saharan Africa.
%
This thesis describes work I contributed to phase 1 of the Ag1000G Project.
%


\end{abstract}


%%%%%%%%%%%%%%%%%%%%%%%%%%%%%%%%%%%%%%%%%%%%%%%%%%%%%%%%%%%%%%%%%%%%%%%%%%%%%%%
%%%%%%%%%%%%%%%%%%%%%%%%%%%%%%%%%%%%%%%%%%%%%%%%%%%%%%%%%%%%%%%%%%%%%%%%%%%%%%%
\section{Malaria vector control in sub-Saharan Africa}


%%
Malaria is an infectious disease caused by eukaryotic parasites of the genus \textit{Plasmodium} and transmitted by blood-feeding mosquitoes of the genus \textit{Anopheles}.
%
The vast majority of malaria occurs in sub-Saharan Africa, with more than 200 million cases and 400,000 deaths annually \citep{WHO2019WMR}.
%
The most effective methods of controlling malaria prevent disease transmission by targeting the mosquito vector (malaria vector control).
%
Two methods of malaria vector control are widely used in public health programmes in Africa: mass distribution of long-lasting insecticidal bednets (LLINs)
\citep{Carnevale2019,Okumu2020} and indoor residual spraying of houses with insecticides (IRS) \citep{WHO2006IRS,Pluess2010,Choi2019}.
%
At the turn of the millenium, a concerted international campaign and partnership was launched with the goal of dramatically increasing the scale and scope of malaria vector control programmes in sub-Saharan Africa \citep{Nabarro1998}.
%
Since then, more than 2 billion LLINs have been distributed in total, exceeding 100 million LLINs annually \citep{AMP2020} with more than 40\% of the population at risk sleeping under an LLIN from 2014 onwards \citep{Bhatt2015,WHO2019WMR}.
%
IRS programmes are more targeted and reserved for focal areas of higher malaria transmission, but are nevertheless substantial, protecting more than 10\% of the population at risk in 2010, although declining to lower levels since \citep{Bhatt2015,WHO2019WMR,Tangena2020}. 
%
The impact of these vector control programmes has been substantial. 
%
The prevalence of \textit{Plasmodium falciparum} malaria infections in Africa halved between 2000 and 2015, averting approximately 663 million clinical cases, of which 68\% can be attributed to LLIN and IRS interventions \citep{Bhatt2015}.
%%


\printbibliography


\end{document}
