\documentclass[a4paper,11pt,abstracton,hidelinks]{scrartcl}
\usepackage{dphil}
\addbibresource{refs.bib}


\title{
Chapter 1. General introduction
}


\author{}


\begin{document}
\renewcommand{\abstractname}{Summary}


\maketitle


%%%%%%%%%%%%%%%%%%%%%%%%%%%%%%%%%%%%%%%%%%%%%%%%%%%%%%%%%%%%%%%%%%%%%%%%%%%%%%%
%%%%%%%%%%%%%%%%%%%%%%%%%%%%%%%%%%%%%%%%%%%%%%%%%%%%%%%%%%%%%%%%%%%%%%%%%%%%%%%
\begin{abstract}


%%
In this chapter I provide an overview of the present situation regarding malaria control in sub-Saharan Africa, the critical role played by large-scale mosquito control programmes, and the current challenges of insecticide resistance.
%
I then introduce whole-genome sequencing and describe its potential role in malaria mosquito surveillance and accelerating efforts towards the development of new mosquito control tools.
%
These factors provided the background and motivation for the establishment of the \textit{Anopheles gambiae} 1000 Genomes (Ag1000G) Project, an international collaboration to study genetic variation among malaria-transmitting mosquitoes collected from natural populations across sub-Saharan Africa.
%
This thesis describes work I contributed to phase 1 of the Ag1000G Project.
%


\end{abstract}


%%%%%%%%%%%%%%%%%%%%%%%%%%%%%%%%%%%%%%%%%%%%%%%%%%%%%%%%%%%%%%%%%%%%%%%%%%%%%%%
%%%%%%%%%%%%%%%%%%%%%%%%%%%%%%%%%%%%%%%%%%%%%%%%%%%%%%%%%%%%%%%%%%%%%%%%%%%%%%%
\section{Malaria vector control in sub-Saharan Africa}


%%
Malaria is an infectious disease caused by eukaryotic parasites of the genus \textit{Plasmodium} and transmitted by blood-feeding mosquitoes of the genus \textit{Anopheles}.
%
The vast majority of malaria occurs in sub-Saharan Africa, with more than 200 million cases and 400,000 deaths annually \citep{WHO2019WMR}.
%
The most effective methods of controlling malaria prevent disease transmission by targeting the mosquito vector (malaria vector control).
%
Two methods of malaria vector control are widely used in public health programmes in Africa: mass distribution of long-lasting insecticidal bednets (LLINs)
\citep{Carnevale2019,Okumu2020} and indoor residual spraying of houses with insecticides (IRS) \citep{WHO2006IRS,Pluess2010,Choi2019}.
%
At the turn of the millenium, a concerted international campaign and partnership was launched with the goal of dramatically increasing the scale and scope of malaria vector control programmes in sub-Saharan Africa \citep{Nabarro1998}.
%
Since then, more than 2 billion LLINs have been distributed in total, exceeding 100 million LLINs annually \citep{AMP2020} with more than 40\% of the population at risk sleeping under an LLIN from 2014 onwards \citep{Bhatt2015,WHO2019WMR}.
%
IRS programmes are more targeted and reserved for focal areas of higher malaria transmission, but are nevertheless substantial, protecting more than 10\% of the population at risk in 2010, although declining to lower levels since \citep{Bhatt2015,WHO2019WMR,Tangena2020}. 
%
The impact of these vector control programmes has been substantial. 
%
The prevalence of \textit{Plasmodium falciparum} malaria infections in Africa halved between 2000 and 2015, averting approximately 663 million clinical cases, of which 68\% can be attributed to LLIN and IRS interventions \citep{Bhatt2015}.
%%


\begin{figure}[t!]
\begin{center}
\includegraphics*[width=1\linewidth,center]{artwork/chapter1/llins.png}
\end{center}
\caption{
%
\textbf{Long-lasting insecticidal bednets (LLINs) distributed in sub-Saharan Africa}.
%
Bar colours denote countries, ordered vertically by the total number of nets distributed since 2000, largest at the bottom.
%
The eight countries with the largest numbers of nets are: Nigeria (dark blue); Democratic Republic of the Congo (light blue); Ethiopia (dark orange); Kenya (light orange); Uganda (dark green); Tanzania (light green); Ghana (dark red); Madagascar (light red).
%
Data from \citet{AMP2020}.
%
}
\label{fig:llins}
\end{figure}


\section{The threat of insecticide resistance}


%%
All LLINs approved for public health use by the World Health Organization (WHO) use a pyrethroid insecticide \citep{Elliott1989} as the primary active ingredient \citep{WHO2020PQVC}. 
%
The majority of countries with IRS programmes have also primarily used pyrethroid insecticides \citep{WHO2019WMR,Tangena2020}.
%
The heavy reliance on a single insecticide class has inevitably led to a substantial increase in the prevalence of resistance to pyrethroid insecticides among African malaria vectors \citep{Hemingway2016}. 
%
Pyrethroid resistance is conventionally monitored via standardised bioassays, and resistance is detected when the proportion of mosquitoes killed by a diagnostic dose falls below a standard threshold \citep{WHO2018TPIRM}. 
%
Many countries routinely perform these insecticide resistance bioassays as part of entomological monitoring programmes. 
%
When these bioassay data have been aggregated, there is a clear trend of increasing pyrethroid resistance since 2000 within the major African malaria vector species \citep{Hancock2020}. 
%
This trend is particularly acute in West Africa where, for example, it is estimated that more than 80\% of the region has resistance to the pyrethroid deltamethrin since 2010. 
%%


%%
Resistance to pyrethroid insecticides in malaria vectors clearly has the potential to reduce the efficacy of LLIN and IRS programmes.
%
In practice, however, it has not been straightforward to demonstrate or measure this reduction in efficacy.
%
A WHO-coordinated observational study in four African countries and India found no evidence that pyrethroid resistance had any effect on malaria infection prevalence or disease incidence in areas using pyrethroid LLINs as the primary intervention \citep{Kleinschmidt2018}.
%
On the other hand, several studies have found that switching IRS programmes from pyrethroids to a different insecticide class has led to a significant reduction in disease \citep{Hargreaves2000,Kafy2017}. 
%
Using non-pyrethroid IRS in addition to pyrethroid LLINs, or using newer LLINs that combine a pyrethroid with the synergist piperonyl butoxide (PBO) that counteracts some forms of pyrethroid resistance, have also been shown to reduce disease relative to pyrethroid-only LLINs alone \citep{Protopopoff2018}. 
%
Mathematical modelling of aggregated data from bioassays and experimental hut trials has also provided evidence that non-pyrethroid IRS substantially reduces malaria relative to pyrethroid IRS \citep{SherrardSmith2018} and that PBO LLINs are more effective than pyrethroid-only LLINs in areas with pyrethroid resistance \citep{Churcher2016}.
%
There is a broad consensus that pyrethroid resistance now poses a significant threat to malaria vector control programmes, and is at least in part responsible for the fact that malaria prevalence has not substantially reduced in sub-Saharan Africa since 2014 and has increased in some countries \citep{Hemingway2016,WHO2019WMR}. 
%%


\printbibliography


\end{document}
