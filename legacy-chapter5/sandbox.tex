%% DK discussion 2019-05-03
molecular evolution - layering of adaptations, epistatic interactions, sequence of events.
%
Population processes - broad geographical and ecological scale of a selective sweep, gives insights into ongoing adaptive processes we wouldn't otherwise have.
%
species complexes, species boundaries
%
interesting molecule across insects
%
chemicals dumped into the environment
%
significant shifts in allele frequency occurring on the timescale of a decade, observable time frame
%
how many other examples in biology of all those things? apart from doing stuff in labs.
%
especially small in eukaryotes, insects (drosophila, aedes, anopheles) and helminths.
%
short abstract (150 words) that brings out the key points - big story, big issue, what was data that we collected, how does it inform the big question, here is the big conclusion
%
three options - (1) submit as-is; (2) restyle the paper; (3) ship as-is, then write a companion paper that makes the big picture, has some cartoons, and shows how the VGSC paper provides an example; like a trends paper, e.g., commentary or opinion paper, or even a data paper with very little more data but adds a little zest.
%
key features of existing paper - dense variation data; overlaid on phenotype knowledge; tried to understand link from genotype to phenotype, emerging lineages. Then if you were an epemiologists, if you're interested in how drug resistance is spreading, we've categorised into a set of types, where are those types, what can we learn about that if we overlaid another 10 years of data, how would it help to manage insecticide resistance better?
%
e.g., rare variant data, don't give it room to breath or to be verified. double jeopardy. so e.g., put rare variant analysis as an interesting illustration to a trends paper (what-if)? or put in as a proposed methodological approach?
%
So begin with writing thesis chapter, then see if that gives substance for restyling of first paper? or maybe enough to pull out for a trends paper? indicate it's a really rich area, and an interest class of genes.
%
What are interesting questions? Material for thesis discussion. Also input for Sanger QQ. Define a set of problems. Need more data of this sort. More methods of this sort. Join up with more data of other types.
%
What we've done already, bring out key concepts, strengthen for journal submission. Then stack up other ideas for final discussion.
%%


%% two binding sites
Molecular modelling studies have predicted two independent pyrethroid binding sites within the channel pore @@cite.
%
However, the significance of these two binding sites, or the exact nature of the interaction between channel and pyrethroid molecules, is still not clear @@keep this sentence?@@.
%%


%% sandbox
%
The high degree of homology between insect VGSC protein sequences means that studies of molecular mechanisms of resistance in any insect species are potentially relevant to other insect species.
%
The molecular biology of pyrethroid resistance in insects is complex, with in excess of @@40 amino acid substitutions shown to functionally affect pyrethroid resistance in vitro, and a further @@N substitutions found to be associated with resistance in vivo @@cite.
%
As a shorthand, any VGSC substitution that causes resistance is referred to as an instance of "target-site resistance", because the VGSC is the binding target of pyrethroid molecules.
%
However, the molecular actions of these substitutions are diverse, and do no necessarily involve protein changes that directly alter the pyrethroid binding site @@cite.
%
A number of substitutions have been found at amino acid sites within one of the two predicted pyrethroid binding sites, and it is reasonable to assume that these act by changing the conformation of the binding site and thus altering the binding interaction.
%
However, many other resistance substitutions have been found that are outside either of the binding sites, e.g. @@example, @@cite.
%
Some studies have found evidence that such substitutions might allosterically alter the conformation of a pyrethroid binding site, despite being located in a relatively distant domain.
%
It has also been suggested that substitutions might alter the dynamic properties of channel gating in some way, and thus indirectly ameliorate the impact of pyrethroid molecules @@cite @@improve.

%%
TODO paragraph on the origins of pyrethroid resistance ... agriculture versus public health use ... geographical origins ... cross-resistance to DDT, pre-adaption?
%%


%%
TODO paragraph on the molecular mechanisms of resistance to pyrethroids in An.  gambiae ... target-site resistance (kdr, N1570Y, mutations in other species) ... metabolic resistance ... increasing population frequency of kdr
%%

%% sandbox
%
Pyrethroid resistance in malaria vectors thus presents both a public health emergency and a phenomenon of great biological interest.
%
Opportunity to observe natural selection in action at the scale of an entire continent, within mosquito species that themselves span a broad range of ecosystems and exhibit complex population structure.
%
Study the factors that lead to increased selective pressure and the conditions that predispose populations to the rapid evolution and fixation of resistance alleles.
%
Reveal the connectivity between populations, we can watch the geographical spread of resistance in the same way that one might watch the spread of an outbreak of an infectious disease with multiple outbreak foci.
%
Also opportunity to observe molecular evolution, within a gene that has been massively conserved across insects, is now rapidly exploring a new molecular landscape of adaptation.
%
Watch the interplay of molecular function and constraint with selective pressure, as mutations accumulate and combine, sending the protein perhaps irreversibly into an entirely new evolutionary trajectory.
%
I.e., not just caused positive selection for a new mutation, but sent this ancient protein down an entirely new evolutionary trajectory.
%%


%% paragraph about opportunities for improved genetic surveillance of resistance?
