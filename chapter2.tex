\documentclass[a4paper,11pt,abstracton,hidelinks]{scrartcl}
\usepackage{dphil}
\addbibresource{refs.bib}


\title{
Chapter 2. Historical context: correspondence on the discovery of the \textit{Anopheles gambiae} species complex
}


\author{}


\begin{document}
\renewcommand{\abstractname}{Summary}


\maketitle


\begin{displayquote}
``Every eradication campaign is, at all stages, a piece of operational research and our ignorance is such that we must be prepared in all cases to learn as we go along. The usefulness of the entomologist in the final phases of an eradication campaign must largely depend on the amount they have been able to learn during the preceding phases and the extent to which they have been willing and able to peer ahead into the future.'' \\\\ Peter Mattingly \citeyearpar{Mattingly1963}
\end{displayquote}



%%%%%%%%%%%%%%%%%%%%%%%%%%%%%%%%%%%%%%%%%%%%%%%%%%%%%%%%%%%%%%%%%%%%%%%%%%%%%%%
%%%%%%%%%%%%%%%%%%%%%%%%%%%%%%%%%%%%%%%%%%%%%%%%%%%%%%%%%%%%%%%%%%%%%%%%%%%%%%%
\begin{abstract}


%%
In this chapter I look back to a series of discoveries that were made during the first global malaria eradication campaign of the 1960s, which uncovered the fact that \textit{Anopheles gambiae} is not a single mosquito species but rather a complex of multiple morphologically-identical species, with important differences in their ecology, behaviour and feeding preferences. 
%
Two entomologists, George Davidson and Hugh Paterson, were at the forefront of these discoveries, and I draw on both the published literature of the time and a collection of previously unpublished letters to tell both the public and private story of their collaboration.
%
These events provide an introduction to our current understanding of the \textit{Anopheles gambiae} complex, which includes the mosquito species responsible for the majority of malaria transmission in sub-Saharan Africa.
%
They also mark the introduction of genetic methods into the operational surveillance of malaria vectors, and provide a valuable perspective on present day challenges in malaria vector control.
%
The full correspondence between Davidson and Paterson is provided as a supplementary file.
%%


\end{abstract}


\tableofcontents


%%%%%%%%%%%%%%%%%%%%%%%%%%%%%%%%%%%%%%%%%%%%%%%%%%%%%%%%%%%%%%%%%%%%%%%%%%%%%%%
%%%%%%%%%%%%%%%%%%%%%%%%%%%%%%%%%%%%%%%%%%%%%%%%%%%%%%%%%%%%%%%%%%%%%%%%%%%%%%%
\section{\textit{Anopheles gambiae} Giles}

%%
The mosquito species \textit{Anopheles gambiae} was first described in 1902 in the second edition of Robert M. Giles' handbook on mosquitoes (@@cite Giles 1902).
%
As with all Anopheline mosquitoes, \textit{Anopheles gambiae} has a life cycle that involves aquatic egg, larval and pupal stages, and an adult stage where both males and females feed on nectar but females also require a blood meal to complete egg development.
%
This blood feeding behaviour provides the opportunity for transmission of parasites between hosts, such as the Plasmodium parasites causing malaria.
%
However, entomologists studying malaria in Africa during the early part of the 20th century discovered that, among the more than one hundred Anopheles mosquito species they encountered, only a handful were able to transmit human malaria, and of those, \textit{Anopheles gambiae} Giles was most often the dominant vector (@@cite de Meillon 1947; de Meillon 1950; Gillies and de Meillon 1968).
%%

%%
Following the advent of chemical insecticides, during the 1940s there were early demonstrations that malaria vectors could be effectively controlled by indoor residual spraying of insecticides.
%
Optimism spread that malaria could be eradicated by eliminating its mosquito vector, and a global malaria eradication programme (GMEP) was launched by the WHO in 1955 (@@cite Najera et al. 2011).
%
However, spraying campaigns met with mixed success in sub-Saharan Africa.
%
Insecticide resistance quickly emerged at several locations where spraying had been carried out, and mosquitoes in some areas appeared to change their behaviour to avoid the insecticides.
%
Two entomologists, George Davidson in London and Hugh Paterson in Johannesburg, were at the forefront of work seeking to understand these events.
%
Learning of each other's work via preliminary reports published by the WHO, Davidson and Paterson established a long-distance collaboration.
%%


%%%%%%%%%%%%%%%%%%%%%%%%%%%%%%%%%%%%%%%%%%%%%%%%%%%%%%%%%%%%%%%%%%%%%%%%%%%%%%%
%%%%%%%%%%%%%%%%%%%%%%%%%%%%%%%%%%%%%%%%%%%%%%%%%%%%%%%%%%%%%%%%%%%%%%%%%%%%%%%
\subsection{George Davidson}

%%
George Davidson was based at the Ross Institute of Tropical Hygiene in London, and was involved during the 1940s in early trials of indoor residual spraying to counter malaria in Sierra Leone, Democratic Republic of Congo, Tanzania and Kenya.
%
In 1954, the Western Sokoto malaria control pilot project in Northern Nigeria began indoor residual spraying of the insecticides DDT and dieldrin (@@cite Bruce-Chwatt and Archibald 1959), but less than two years later the first reports emerged of resistance to these insecticides among anopheline mosquitoes in sprayed regions (@@cite Elliott and Ramakrishna 1956).
%
Eggs of resistant and susceptible mosquitoes were sent to Davidson in London, where they were used to establish mosquito colonies and study resistance under controlled conditions.
%
Davidson confirmed that mosquitoes from Western Sokoto were indeed resistant, particularly to dieldrin, which had been sprayed in approximately half of the study area (@@cite Davidson 1956).
%%

%%
Davidson and colleagues continued to investigate the genetic basis of dieldrin resistance, using crosses between resistant and susceptible parents from colonies of mosquitoes identified as \textit{Anopheles gambiae} Giles originating from different locations throughout Africa.
%
As they did so, they began to find that crosses between certain pairs of colonies always yielded male offspring that were unable to reproduce, although the female offspring were fully fertile.
%
Recognising sterility in one sex as a hallmark of interbreeding between different species (@@cite Haldane 1922), Davidson and colleagues began to suspect that \textit{Anopheles gambiae} Giles may in fact comprise more than one species.
%%


%%%%%%%%%%%%%%%%%%%%%%%%%%%%%%%%%%%%%%%%%%%%%%%%%%%%%%%%%%%%%%%%%%%%%%%%%%%%%%%
%%%%%%%%%%%%%%%%%%%%%%%%%%%%%%%%%%%%%%%%%%%%%%%%%%%%%%%%%%%%%%%%%%%%%%%%%%%%%%%
\subsection{Hugh Paterson}

%%
At the time that Davidson was performing crosses in London, Hugh Paterson was engaged as a WHO consultant at the East African Institute of Malaria and Vector-borne Diseases in Amani, Tanzania.
%
Anopheles larvae are generally found in fresh water only, but reports had emerged from East Africa of \textit{Anopheles gambiae} Giles mosquitoes able to breed in estuarine salt-water environments (@@cite de Meillon 1947; Muirhead-Thomson 1951; and refs in Paterson 1962), and Paterson took the opportunity to investigate.
%
He performed crosses between three colonies, finding male sterility and other evidence of reproductive incompatibility between fresh-water and salt-water-breeding parents.
%%

%%
After completing his tenure in Tanzania, Paterson joined Peter Mattingly, an entomologist and taxonomist from the British Museum, on a tour of several countries in Southern Africa including Swaziland, Mozambique, Zimbabwe and Mauritius.
%
The purpose of the tour was to investigate reports alleging that \textit{Anopheles gambiae} mosquitoes had changed their behaviour in response to insecticide spraying campaigns, preferring to feed outdoors instead of indoors in order to avoid contact with insecticides (@@cite Mattingly 1963).
%
Paterson collected eggs whilst on his travels and, on his return to his permanent position at the South African Institute for Medical Research in Johannesburg, used them to establish colonies and conduct crossing experiments.
%
These crosses, like those being performed by Davidson in London, suggested that the status of \textit{Anopheles gambiae} Giles as a single biological species needed to be reconsidered.
%%


%%%%%%%%%%%%%%%%%%%%%%%%%%%%%%%%%%%%%%%%%%%%%%%%%%%%%%%%%%%%%%%%%%%%%%%%%%%%%%%
%%%%%%%%%%%%%%%%%%%%%%%%%%%%%%%%%%%%%%%%%%%%%%%%%%%%%%%%%%%%%%%%%%%%%%%%%%%%%%%
\section{1962: Species A and B}

%%
Davidson first described the results of his crossing experiments in a WHO/Mal report in 1962.
%
The report entitled ``Incipient speciation in \textit{Anopheles gambiae} Giles'', 
provided the first evidence for two distinct fresh-water forms of \textit{Anopheles gambiae}, which Davidson named ``Group A'' and ``Group B'' (@@cite Davidson and Jackson 1962).
%
A draft of this paper reached Paterson whilst on tour, and he wrote to Davidson from Mauritius in March 1962:
%
\begin{displayquote}
``I have just received your recent WHO/Mal report and have read it with the greatest interest.
%
I have also had the privilege this week of discussing the work we have been doing on gambiae with Peter Mattingly.
%
May I say that I agree with you that your evidence is best interpreted as indicating that your two groups are separate species, especially when one takes Coronel's evidence of the presence of both at Diggi.
%
I believe it will be shown that at Muheza, Tanganyika, both forms are present.'' (Letter 1)
\end{displayquote}
%%


%%
Patersons' own work on crosses between fresh and salt water-breeding mosquitoes in Tanzania also first appeared as a WHO/Mal report (@@cite Paterson 1962), and Davidson replied:
%
\begin{displayquote}
``Thank you for your letter of March 9th last. 
%
I had read the account of your work at Amani with great interest [...] I think our strains, which are all fresh-water strains, are probably more closely-related than were your fresh-water and salt-water strains.
%
I am now trying to get eggs of Melas from Liberia, and of salt-water tolerant gambiae from Amani, to try some crosses here in London.'' (Letter 2)
\end{displayquote}
%%


%%
Davidson's letter also discussed the search for morphological characteristics which could be used to distinguish the different species.
%
Such characteristics were highly desirable because, if they could be found, they would provide a means of identifying mosquitoes in the field, rather than having to transport them to the lab and identify them via cross-mating with colonies of known types.
%
This was highly relevant to the ongoing malaria eradication campaigns of the time, where field entomologists needed practical methods to investigate issues such as insecticide resistance and mosquito behaviour, which could differ between species.
%
However, early studies suggesting differentiating characteristics between group A and group B (@@cite Coronel 1962) proved to be premature, as Davidson wrote:
\begin{displayquote}
``I have been looking through Miss Coronel's detailed figures on sector spot measurements and find considerable differences between strains within a group.
%
[...] We have just identified a recently acquired gambiae strain from the Ivory Coast as belonging to group B from sector spot measurement and pupal spine characters, whereas from crossings it obviously belongs to group A.
%
Therefore the morphological method may not always be reliable.'' (Letter 2)
\end{displayquote}
%%


%%
Davidson followed up Paterson's work, performing crosses between the two freshwater groups A and B and salt-water tolerant strains from both East and West Africa, publishing the results later that year, confirming male sterility between all four forms (@@cite Davidson 1962).
%
For Davidson, however, the fact that sterility was restricted only to males, and therefore that there remained the possibility of gene flow between these different forms, created doubt as to whether they should be considered separate species.
%%


%%
By May of 1962, Paterson had completed his tour and returned to Johannesburg.
%
Paterson wrote to Davidson to discuss his results, and his letter illustrates one of the technical challenges that both entomologists were confronting at the time.
%
Repeating each other's experiments required establishing and maintaining a consistent set of reference colonies between two different research institutes, but achieving that was a logistical challenge, not least because establishing colonies required posting of mosquito eggs between two continents.
%
Paterson wrote:
\begin{displayquote}
``I would like to suggest to you that all of us working on gambiae complex studies should standardize our reference colonies.
%
I should like to see the Kisumu colony used as the ``type'' of group A since it is the most widely used colony of A. gambiae.
%
I hope you will agree with this and suggest a colony as a reference colony for group B. 
%
[...] I should be very pleased if you would help us by sending us eggs of the group B colony you choose so that we may establish it here.'' (Letter 3)
\end{displayquote}
%%

%%
Davidson wrote back in June to discuss the issue of establishing reference colonies, and with further news on the search for distinguishing morphological characteristics, which continued to be fruitless (Letter 4).
%
Paterson responded at the end of June, acknowledging successful receipt and hatching of eggs from group B sent by Davidson.
%
He also shared findings from experiments using colonies established from eggs obtained in Mauritius:
\begin{displayquote}
``I have now found evidence of both group A and group B on Mauritius. [...] 
%
This is very interesting in any case since it is one more spot where group A and group B are sympatric.
%
Of course, what I assume to be group B may be a new group, but I think this is unlikely.'' (Letter 5)
\end{displayquote}
The evidence that two reproductively isolated groups could be found living together at the same location (sympatric) was the key criterion for Paterson to support the elevation of these groups to species rank, because it would establish that although gene flow was possible under lab conditions, in nature the two groups remained genetically distinct.
%%


%%
In the final letter from 1962, Paterson wrote in July with some intriguing results:
\begin{displayquote}
``I obtained some curious results with a cross I made the other day. I managed to get a couple of wild caught gambiae from Southern Rhodesia to lay some eggs. From these we got adults which we set up in both directions against Kisumu adults. The results obtained differ strikingly from results obtained from crosses between groups A \& B. [...] Although one cannot base much on an isolated case like this it does suggest that there may be yet another member of the freshwater complex.'' (Letter 7)
\end{displayquote}
%
Here was the first hint of a further species discovery in Southern Africa.
%%


%%%%%%%%%%%%%%%%%%%%%%%%%%%%%%%%%%%%%%%%%%%%%%%%%%%%%%%%%%%%%%%%%%%%%%%%%%%%%%%
%%%%%%%%%%%%%%%%%%%%%%%%%%%%%%%%%%%%%%%%%%%%%%%%%%%%%%%%%%%%%%%%%%%%%%%%%%%%%%%
\section{1963: Species C}


@@TODO


\printbibliography


\end{document}
