\documentclass[a4paper,11pt,abstracton,hidelinks]{scrartcl}

\usepackage[margin=3cm]{geometry}
\usepackage{graphicx}
\usepackage[UKenglish]{babel}
\usepackage{csquotes}
\usepackage[style=authoryear,citestyle=authoryear,backend=biber,sorting=none,doi=false,url=false]{biblatex}
\usepackage{float}
\usepackage[export]{adjustbox}
\usepackage[T1]{fontenc}
\usepackage{lmodern}
\usepackage[textsize=tiny]{todonotes}
\usepackage[labelsep=period,font=small,labelfont=bf,format=plain]{caption}
\captionsetup[table]{
  position=above,
  belowskip=10pt,
  aboveskip=0pt,
}
\usepackage[group-separator={,}]{siunitx}
\usepackage{booktabs}
\usepackage{pdflscape}
\usepackage{tablefootnote}
\usepackage{authblk}
\usepackage{threeparttable}
\usepackage{afterpage}
\usepackage{lineno}
\linenumbers
\usepackage{setspace}
\usepackage{hyperref}
\doublespacing

\newcommand{\beginsupplement}{%
  \setcounter{table}{0}
  \renewcommand{\thetable}{S\arabic{table}}%
  \setcounter{figure}{0}
  \renewcommand{\thefigure}{S\arabic{figure}}%
}


\addbibresource{../refs.bib}


\title{Chapter 5. The evolution and spread of target-site resistance to pyrethroid insecticides}


\author[1,2]{\small Alistair Miles}
\affil[1]{\footnotesize Big Data Institute, University of Oxford, Li Ka Shing Centre for Health Information and Discovery, Old Road Campus, Oxford OX3 7LF}
\affil[2]{\footnotesize Wellcome Sanger Institute, Hinxton, Cambridge CB10 1SA}


\begin{document}

\maketitle


\begin{abstract}


%%
TODO
%%


\end{abstract}


\section*{Introduction}


%%
Pyrethroids are a class of synthetic insecticides, based on a natural compound pyrethrin found in the flowers of pyrethrum (\textit{Chrysanthemum}) plants \parencite{Elliott1989}.
%
Pyrethroids suitable for commercial use in agriculture and public health were discovered in the 1970s, including permethrin \parencite{Elliott1973} and deltamethrin \parencite{Elliott1974}.
%
These compounds have a high toxicity to insects but relatively low toxicity to mammals, and are photostable but do not accumulate to contaminate the environment.
%
Pyrethroids were approved for use in public health by the WHO pesticide evaluation scheme (WHOPES) in @@year @@cite.
%
Initially they were proposed for using in space and residual spraying, however it was also found that they could be impregnated into bed-nets @@cite.
%
Several landmark clinical trials during the 1980s showed that the use of pyrethroid-treated bed-nets caused a significant reduction in malaria @@cite.
%
Advances in net manufacturing subsequently allowed the development of long-lasting insecticidal nets (LLINs) which retained insecticidal activity for up to 3 years without retreatment @@cite.
%
Pyrethroid LLINs thus became the cornerstone of efforts to scale up malaria control interventions in Africa, led by the Roll Back Malaria Consortium, beginning in 1998 @@cite.
%
Those efforts have continued to expand, and in 2018, 200 million pyrethroid LLINs were distributed for free in Africa, by the Global Fund and other international partner organisations @@cite.
%
The widespread use of pyrethroid LLINs has undoudtedly had a major impact on malaria burden, with modelling studies estimating that LLINs account for @@what fraction of the reduction in malaria cases between 2000 and 2015.
%%


%% paragraph introducing resistance to Pyrethroids
TODO
%%


%%
Pyrethroids disrupt the insect nervous system, causing paralysis ("knock-down") and then death @@cite.
%
Pyrethroid molecules interact with the voltage-gated sodium channel (VGSC), an essential protein which propagates nerve impulses via action potentials @@cite.
%
The precise nature of the interaction, and its effect on the function of the VGSC protein, has not been fully characterised @@cite.
%
However, molecular modelling studies have predicted two independent pyrethroid binding sites within the channel pore @@cite.
%
In the presence of a pyrethroid, VGSC proteins @@describe physiological effect, as shown via oocyte experiments @@cite, suggesting @@what the gating function is disrupted and resting potential cannot be restored.
%
The VGSC protein sequence of \textit{Anopheles gambiae}, the major vector of malaria in Africa, comprises @@N amino acids and has a high degree of homology with sequences from other insect species @@cite.
%
Transcriptional variation has been found, with several exons being optional or having optional extensions, again similar to other insects @@cite.
%%


%%%%%%%%%%%%%%%%%%%%%%%%%%%%%%%%%%%%%%%%%%%%%%%%%%%%%%%%%%%%%%%%%%%%%%%%%%%%%%%
%%%%%%%%%%%%%%%%%%%%%%%%%%%%%%%%%%%%%%%%%%%%%%%%%%%%%%%%%%%%%%%%%%%%%%%%%%%%%%%


\section*{Results}


%%%%%%%%%%%%%%%%%%%%%%%%%%%%%%%%%%%%%%%%%%%%%%%%%%%%%%%%%%%%%%%%%%%%%%%%%%%%%%%
\subsection*{TODO}


%%
TODO
%%


%%%%%%%%%%%%%%%%%%%%%%%%%%%%%%%%%%%%%%%%%%%%%%%%%%%%%%%%%%%%%%%%%%%%%%%%%%%%%%%
%%%%%%%%%%%%%%%%%%%%%%%%%%%%%%%%%%%%%%%%%%%%%%%%%%%%%%%%%%%%%%%%%%%%%%%%%%%%%%%
\section*{Discussion}


%%%%%%%%%%%%%%%%%%%%%%%%%%%%%%%%%%%%%%%%%%%%%%%%%%%%%%%%%%%%%%%%%%%%%%%%%%%%%%%
\subsection*{TODO}


%%
TODO
%%


%%%%%%%%%%%%%%%%%%%%%%%%%%%%%%%%%%%%%%%%%%%%%%%%%%%%%%%%%%%%%%%%%%%%%%%%%%%%%%%
%%%%%%%%%%%%%%%%%%%%%%%%%%%%%%%%%%%%%%%%%%%%%%%%%%%%%%%%%%%%%%%%%%%%%%%%%%%%%%%
\section*{Methods}


%%%%%%%%%%%%%%%%%%%%%%%%%%%%%%%%%%%%%%%%%%%%%%%%%%%%%%%%%%%%%%%%%%%%%%%%%%%%%%%
\subsection*{TODO}

%%
TODO
%%


%%%%%%%%%%%%%%%%%%%%%%%%%%%%%%%%%%%%%%%%%%%%%%%%%%%%%%%%%%%%%%%%%%%%%%%%%%%%%%%
%%%%%%%%%%%%%%%%%%%%%%%%%%%%%%%%%%%%%%%%%%%%%%%%%%%%%%%%%%%%%%%%%%%%%%%%%%%%%%%
\printbibliography


\beginsupplement


%%%%%%%%%%%%%%%%%%%%%%%%%%%%%%%%%%%%%%%%%%%%%%%%%%%%%%%%%%%%%%%%%%%%%%%%%%%%%%%
%%%%%%%%%%%%%%%%%%%%%%%%%%%%%%%%%%%%%%%%%%%%%%%%%%%%%%%%%%%%%%%%%%%%%%%%%%%%%%%
\section*{Supplementary figures}

\clearpage


%%%%%%%%%%%%%%%%%%%%%%%%%%%%%%%%%%%%%%%%%%%%%%%%%%%%%%%%%%%%%%%%%%%%%%%%%%%%%%%
%%%%%%%%%%%%%%%%%%%%%%%%%%%%%%%%%%%%%%%%%%%%%%%%%%%%%%%%%%%%%%%%%%%%%%%%%%%%%%%
\section*{Supplementary tables}

\clearpage


\end{document}
